\section{Wstęp}
    \subsection{Założenia}
        \begin{itemize}
            \item Pomiar prądu rzędu $10\ nA$,
            \item Wykorzystanie ditheringu szumem Gaussowskim, 
            \item Wykorzystanie mikroprocesora STM32F103C8T6 z 12 bitowym ADC,
            \item Wykorzystanie wzmacniacza pomiarowego INA333.
        \end{itemize}
    \subsection{Schemat Blokowy}
    \begin{figure}[!ht]
        \centering
        \scalebox{1}{\begin{subfigure}{\textwidth}
    \hspace{0cm}
    \begin{tikzpicture}
        \draw
            (0, 0) node[op amp](amp1){}
            (0, -3) node[draw, rectangle, minimum width = 2cm, minimum height = 2cm, label = {[align = center]center:$V_{REF}$}](REF){}
            (3, -3) node[draw, rectangle, minimum width = 2cm, minimum height = 2cm, label = {[align = center]center:Noise \\Gen}]{}
            (6, -1) node[adder] (add) {}
            (10, -1) node[draw, rectangle, minimum width = 3cm, minimum height = 5cm, label = {[align = center]center:MCU}](MCU){}
            (10, -6) node[draw, rectangle, minimum width = 2cm, minimum height = 2cm, label = {[align = center]center:USB}](USB){}
            (5, -6) node[draw, rectangle, minimum width = 2cm, minimum height = 2cm, label = {[align = center]center:PSU}](PSU){}
            % (add.135) ++ (-1, 1) -- ++ (0.5, 0) -- (add)
            (add.225) ++ (-1.65, -1) -- ++ (1, 0) -- (add) 
            (add) -- ++ (2.5, 0)

            (amp1) ++ (0.3, -0.3) -- ++ (0, -1.7)
            (amp1.out) -- ++ (4.2, 0) -- (add)
            (amp1.+) -- ++ (-1, 0) to[/tikz/circuitikz/bipoles/length=20pt, R, l=$R_{meas}$] ++ (0, 1)
            (amp1.-) -- ++ (-1, 0) to[short, -o] ++ (0, 1) node[above]{$in-$}
            (amp1.+) ++ (-1, 0) to[short, -o] ++ (0, -1) node[below]{$in+$}
            (USB) ++ (0.5, 1) -- ++ (0, 1.5) 
            (USB) ++ (-0.5, 1) -- ++ (0, 1.5) 
            (PSU) ++ (1, 0) -- ++ (3, 0)
        ;
            % wzmacniacz pomiarowy, zrodlo ref, generacja szumu, sumator, MCU, USB, PSU
    \end{tikzpicture}
\end{subfigure}}
        \caption{Schemat blokowy układu do pomiaru natężenia prądu.}
        \label{sch:BD}
    \end{figure}
    \begin{figure}[!ht]
        \centering
        \scalebox{1}{\begin{subfigure}{\textwidth}
    \hspace{3cm}
    \begin{tikzpicture}
        \draw
            (0, 0) node[op amp, scale = 2](amp){}
            (amp.up) to[short, -o] ++ (0, 0.5) node[above]{$3.3\ V$}
            (amp.down) node[ground]{}
            
            (amp.+) to[R, l=$R_{4}$, a=$1.2k$] ++ (-2, 0) to[short, -o] ++ (0, 0.5) node[above]{$3.3\ V$} 
            (amp.+) ++ (0, -2.5) node[ground]{} to[R, -*, l=$R_{5}$, a=$1k$] ++ (0, 2.5)
            
            (amp.-) to[R, -o, l=$R_{3}$] ++ (-5.2, 0) node[above]{$V_{noise}$}

            (amp.out) -- ++ (0, 3) to[R, l=$R_{2}$, a=$10k$] ++ (-5, 0) coordinate(test)
            to[R, l=$R_{1}$, a=$10k$] ++ (-5, 0) to[short, -o] ++ (0, 0) node[above]{$V_{in}$}
            (test) to[short, -*] ++ (0, -2)
            (amp.out) to[short, -o] ++ (1, 0) node[above]{$V_{out}$}
        ;
    \end{tikzpicture}
\end{subfigure}}
        \caption{Układ sumacyjny dla dodania sygnału pomiarowego i szumu. }
        \label{sch:sumator}
    \end{figure}
    \begin{figure}[!ht]
        \centering
        \scalebox{1}{    \begin{figure}[H]
        \centering
        \includegraphics[width=1\linewidth]{images/psu_tran.png}
        \caption{Symulacja tran stabilizatora napięcia LM1117-3.3 napięcia wejściowego (Vin) i wyjściowego(Vout). }
    \end{figure}}
        \caption{Układ stabilizacji napięcia na 3.3V. }
        \label{sch:psu}
    \end{figure}
\section{Symulacje układów}
    Podczas symulacji sprawdzono działanie kilku bloków funkcjonalnych, tj. PSU, generator szumu białego oraz 
    wzmacniacz pomiarowy. Symulacja czasowa PSU pozwoliła sprawdzić jak duży będzie przerzut napięcia podczas uruchomiania 
    układu. Wyniki przedstawiono na rysunku \ref{fig:sym_LM1117}.   
    \begin{figure}[!ht]
        \centering
        \includegraphics[width = \textwidth]{psu_tran.png}
        \caption{Symulacja tran stabilizatora napięcia LM1117-3.3 napięcia wejściowego ($V_{in}$) i wyjściowego ($V_{out}$).}
        \label{fig:sym_LM1117}
    \end{figure}
    % \begin{figure}[!ht]
    %     \centering
    %     \includegraphics[width = \textwidth]{sumator_dc.png}
    %     \caption{Symulacja dc układu sumującego, napięcia wyjściowego ($V_{out}$) od napięcia wejściowego ($V_{in}$).}
    %     \label{fig:sym_sum}
    % \end{figure}
    
    Podczas symulacji wzmacniacza pomiarowego okazało się, że możliwe będzie zmierzenie prądów o natężeniu $1\ nA$. Czułość 
    wzmacniacza określono na $\frac{1\ mV}{1\ nA}$. Zakres pomiarowy wynosi $\pm 1\ \mu A$, a dopuszczalny zakres napięcia 
    wspólnego, przy maksymalnym prądzie wynosi $\approx 0.7\ V \div \approx 2.8\ V$. Zależność napięcia wyjściowego 
    od prądu wejściowego przedstawiono na wykresach \ref{fig:sym_INA_100n} i \ref{fig:sym_INA_1u}. 
    Zależność napięcia wyjściowego od wejściowego napięcia wspólnego dla prądu $1\ \mu A$ została przedstawiona na wykresie 
    \ref{fig:sym_INA_CM_follower}. 
    % Schemat symulacyjny przedstawiono na rysunku \ref{fig:sym_CM_sch}.
    \begin{figure}[!ht]
        \centering
        \includegraphics[width = \textwidth]{INA333_1u_range.png}
        \caption{Symulacja parametryczna wzmacniacza pomiarowego w zakresie $\pm 1\ \mu A$.}
        \label{fig:sym_INA_1u}
    \end{figure}
    \begin{figure}[!ht]
        \centering
        \includegraphics[width = \textwidth]{INA333_100n_range.png}
        \caption{Symulacja parametryczna wzmacniacza pomiarowego w zakresie $\pm 100\ nA$.}
        \label{fig:sym_INA_100n}
    \end{figure}
    % \begin{figure}[!ht]
    %     \centering
    %     \includegraphics[width = \textwidth]{INA333_p1u_cm.png}
    %     \caption{Symulacja dc wejściowego napięcia wspólnego wzmacniacza pomiarowego dla $I_{meas} = 1\ \mu A$, różnica 
    %     napięć $\approx 120\ mV$.}
    %     \label{fig:sym_INA_CM}
    % \end{figure}
    % \clearpage
    \begin{figure}[!ht]
        \centering
        \includegraphics[width = \textwidth]{INA333_p1u_cm_follower.png}
        \caption{Symulacja dc wejściowego napięcia wspólnego wzmacniacza pomiarowego dla $I_{meas} = 1\ \mu A$, różnica 
        napięć $\approx 65\ \mu V$.}
        \label{fig:sym_INA_CM_follower}
    \end{figure}

    \begin{figure}[!ht]
        \centering
        \includegraphics[width = \textwidth]{INA333_K_diff.png}
        \caption{Wzmocnie różnicowe INA333.}
        \label{fig:INA333_Kd}
    \end{figure}
    \begin{figure}[!ht]
        \centering
        \includegraphics[width = \textwidth]{INA333_K_ref.png}
        \caption{Wzmocnienie od wejścia \textit{REF} wzmacniacza pomiarowego.}
        \label{fig:INA333_Kref}
    \end{figure}
    \begin{figure}[!ht]
        \centering
        \includegraphics[width = \textwidth]{Noise_gen_out.png}
        \caption{Charakterystyki częstotliwościowe układu: czerwona - pasmo przenoszenia generatora szumu, zielona - 
        pasmo przenoszenia na wyjściu INA333, niebieska - pasmo przenoszenia za filtrem LPF z wtórnikiem - UWAGA, bez wtónika 
        jest podbicie o $\approx 7\ dB$ na wyjściu INA333 przy $f \approx 150\ kHz$.}
        \label{fig:frequency_charakteristics}
    \end{figure}
    \begin{figure}[!ht]
        \centering
        \includegraphics[width = \textwidth]{INA333_tran_10m_fft_noise.png}
        \caption{Transformata fouriera z $10\ ms$ symulacji transient wzmacniacza pomiarowego z generacją szumu na referencji. 
        Okno fft $100\ Hz$, co daje $\approx 1.65\ \frac{\mu V}{\sqrt{Hz}}$.}
        \label{fig:INA333_Noise_fft}
    \end{figure}
    \begin{table}[!ht]
        \centering
        \begin{tabular}{|c|c|c|}\hline
            parametr & wartość & komentarz \\\hline
            czułość & $1\ \frac{mV}{nA}$ & czułość układu bez ditheringu \\\hline
            zakres pomiarowy & $\pm 1\ \mu A$ & - \\\hline
            zakres napięcia wspólnego & $0.7 \div 2.8\ V$ & - \\\hline
            częstotliwość graniczna & $30\ Hz$ & ograniczenie od INA333 \\\hline
            pasmo szumu białego & $100\ Hz \div 200\ kHz$ & górne ograniczenie od wejścia \textit{REF} INA333 \\\hline
            napięcie szumu & $\approx 0.74\ mV$ & w paśmie $100\ Hz \div 200\ kHz$ \\\hline
            widmowa gęstość szumu & $\approx 1.65\ \frac{\mu V}{\sqrt{Hz}}$ & - \\\hline
            częstotliwość próbkowania & $f_S = 1\ MHz$ & z dokumentacji STM32F103C8T6 \\\hline
            częstotliwość nadpróbkowania & $f_{OVS} = 62.5\ kHz$ & nadpróbkowanie 16 razy \\\hline
            dodatkowe bity & $+2$ bity & na podstawie AN5537 \\\hline
            teoretyczna wartość mierzalnego & \multirow{2}{*}{$I_{meas} \approx 250\ pA$} & \multirow{2}{*}{jeśli nie tracimy na ENOB} \\ 
            prądu przy ditheringu  & & \\\hline
        \end{tabular}
    \end{table}

    % \clearpage
    % \subsection{Problem $V_{CM}$ - do przedyskutowania}
    %     Po przeprowadzeniu dodatkowych symulacji, zauważono, że napięcie sumacyjne może powodować znaczne 
    %     upływności przez rezystor polaryzujący wejścia układu pomiarowego. Schemat symulacyjny przedstawiono 
    %     na rysunku \ref{fig:Vcm_sch}. Podczas symulacji założono mierzony prąd $I_{meas} = 330\ nA$ i modyfikowano 
    %     rezystory dzielnika napięcia w zakresie $0 \div 10\ M\Omega$. Wyniki symulacji przedstawiono na rysunku 
    %     \ref{fig:Vcm_sim}. Różnica napięcia wyjściowego w zakresie dopuszczalnego napięcia wspólnego wyniosła 
    %     $\approx 100\ mV$, co odpowiada różnicy natężenia prądu $100\ nA$, z tym, że dla napięć większych niż 
    %     $V_{REF}$, ta różnica wynosi maksymalnie $17\ mV$. Propozycją rozwiązania problemu jest pomiar 
    %     napięcia sumacyjnego i software'owa kompensacja związanych z nim upływności, jednak taka metoda pozwoli 
    %     prawidłowo mierzyć prądy rzędu setek $nA$. 
    %     \begin{figure}[!ht]
    %         \centering
    %         \includegraphics[width = 0.7\textwidth]{Vcm_sch.png}
    %         \caption{Schemat układu symulacyjnego do kompensacji pradów upływu poprzez pomiar $V_{CM}$.}
    %         \label{fig:Vcm_sch}
    %     \end{figure}
    %     \clearpage
    %     \begin{figure}[!ht]
    %         \centering
    %         \includegraphics[width = \textwidth]{Vcm_sim.png}
    %         \caption{Wyniki symulacji $V_{CM}$ dla prądu $330\ nA$, można zauważyć upływność przez rezystor $R_8$.}
    %         \label{fig:Vcm_sim}
    %     \end{figure}

    %     Drugim podejściem do problemu jest zmiana napięcia wspólnego podawanego przez rezystor do wejść 
    %     wzmacniacza pomiarowego, tak by spadek napięcia na rezystorze był na tyle mały, że prąd upływu 
    %     będzie można zaniedbać. Osiągnięto to poprzez dołożenie wtórnika napięciowego i podłączenie jego wyjścia 
    %     do rezystora. Wtedy spadek napięcia na rezystorze jest bliski napięciu niezrównoważenia wtórnika. 
    %     Schemat układu przedstawiono na rysunku \ref{fig:Vcm_sch_v2}. 
    %     Taka modyfikacja napięcia wspólnego nie powinna wpłynąć na pomiar, ze względu na duży CMRR, jednocześnie 
    %     umożliwiając przepływ prądów polaryzacji, wynoszących według dokumentacji maksymalnie $\pm 200\ pA$. 
    %     Wyniki symulacji przedstawiono na rysunku \ref{fig:Vcm_sim_v2}. Jeśli wejścia nie są nigdzie podłączone, czyli 
    %     nie ma zewnętrznego wymuszenia $V_{CM}$, napięcie wspólne silnie zależy od $V_{OS}$ wtórnika i osiąga górną granicę 
    %     dopuszczalnego zakresu napięć wspólnych, w przypadku $V_{OS} = 25\ mV$, a przy $V_{OS} = -200\ \mu V$, 
    %     $V_{CM} \approx 1.59\ V$, co jest bliskie $V_{REF}$. Z powyższych rozważań wynika, że wtórnik powinien mieć 
    %     małe napięcie niezrównoważenia. 
    %     \begin{figure}[!ht]
    %         \centering
    %         \includegraphics[width = 0.7\textwidth]{Vcm_sch_v2.png}
    %         \caption{Schemat układu symulacyjnego do kompensacji prądów upływu po stronie analogowej.}
    %         \label{fig:Vcm_sch_v2}
    %     \end{figure}
    %     \clearpage
    %     \begin{figure}[!ht]
    %         \centering
    %         \includegraphics[width = \textwidth]{Vcm_sim_v2.png}
    %         \caption{Wyniki symulacji $V_{CM}$ dla prądu $330\ nA$ z kompensacją, upływność przez $R_8$ wynosi $\approx 2.5\ nA$, 
    %         przy $V_{OS}$ wtórnika wynoszącym $25\ mV$.}
    %         \label{fig:Vcm_sim_v2}
    %     \end{figure}
    
    % \begin{figure}[!ht]
    %     \centering
    %     \includegraphics[width=0.7\textwidth]{INA333_cm_sim.png}
    %     \caption{Schemat do symulacji $V_{CM}$.}
    %     \label{fig:sym_CM_sch}
    % \end{figure}

    % \clearpage
    % \subsection{Alternatywne podejście do układu}
    %     \begin{figure}[!ht]
    %         \centering
    %         \includegraphics[width = 0.7\textwidth]{Noise_gen_sch.png}
    %         \caption{Schemat układu generacji szumu wzmacniaczem TSV992.}
    %         \label{fig:sch_noise_gen}
    %     \end{figure}
    %     \begin{figure}[!ht]
    %         \centering
    %         \includegraphics[width = \textwidth]{Noise_gen_output.png}
    %         \caption{Widmowa gęstość szumu na wyjściu układu generacji szumu.}
    %         \label{fig:noise_gen_out}
    %     \end{figure}
    %     \begin{figure}[!ht]
    %         \centering
    %         \includegraphics[width = 0.7\textwidth]{INA_noise_gen.png}
    %         \caption{Wzmacniacz pomiarowy z szumem doprowadzonym do wejścia \textit{REF}.}
    %         \label{fig:sch_INA333_noise}
    %     \end{figure}
    %     \begin{figure}[!ht]
    %         \centering
    %         \includegraphics[width = \textwidth]{INA_noise.png}
    %         \caption{Widmowa gęstość szumu na wyjściu wzmacniacza pomiarowego z szumem dodanym do referencji.}
    %         \label{fig:INA333_output_noise}
    %     \end{figure}
    %     \clearpage
    %     \begin{figure}[!ht]
    %         \centering
    %         \includegraphics[width = \textwidth]{INA_noise_zoom.png}
    %         \caption{Zbliżenie na "interesujący" fragment charakterystyki gęstości szumu wzmacniacza pomiarowego.}
    %         \label{fig:INA333_output_n_zoom}
    %     \end{figure}

\section{Schematy ideowe układu}
    Poniżej przedstawiono schematy części analogowej układu. Podczas pomiarów okazało się, że szum INA333 jest 
    znacznie większy od szumu generowanego przez generator szumu. W związku z tym, zredukowano szum generowany przez 
    generator szumu oraz zmniejszono wzmocnienie INA333 z $Ku = 10000\ \frac{V}{V}$ do $Ku = 100\ \frac{V}{V}$. 
    Modyfikacja wiązała się z wlutowaniem kondensatora o wartości $1\ \mu F$ w sprzężenie zwrotne wyjściowego 
    wzmacniacza generatora szumu, co dało obcięcie pasma do około $100\ Hz$, oraz zmiana rezystora ustalającego 
    wzmocnienie z $10\ \Omega$ do $1\ k\Omega$. Wprowadzone modyfikacje pozwoliły zmniejszyć szum do około 
    $1 \div 2$ LSB. Schemat wzmacniacza pomiarowego przedstawiono na rysunku \ref{sch:INA333}, a generatora 
    szumu na rysunku \ref{sch:Vref_gen}. 
    Wyjście INA333 podłączono do filtra anty-aliasing'owego, który został przedstawiony na rysunku \ref{sch:LPF}.
    Rezystor $R_4$ zgodnie z dokumentacją zastosowanego wzmacniacza pozwala zmniejszyć wpływ obciążenia pojemnościowego na wzmacniacz, 
    ale wprowadza tłumienie sygnału, dlatego zamiast wtórnika zastosowano wzmacniacz w konfiguracji nieodwracającej o niewielkim 
    wzmocnieniu.
    Parametry robocze układu przedstawiono w tabeli \ref{tab:params}.
    \begin{figure}[!ht]
        \centering
        \scalebox{1}{\begin{subfigure}{\textwidth}
    \hspace{3cm}
    \begin{tikzpicture}
        \draw
            (0, 0) node[op amp, scale = 2](amp){} node[]{INA333}
            (amp) ++ (-1.66, -0.5) -- ++ (-1.22, 0) to[/tikz/circuitikz/bipoles/length=30pt, R, l=$R_G$, a=$10\ \Omega$] ++ (0, 1) -- ++ (1.22, 0)
            (amp.up) to[short, -o] ++ (0, 0.5) node[above]{$3.3\ V$}
            (amp.down) node[ground]{}
            (amp.+) -- ++ (-2.75, 0) to[R, l=$R_{meas}$, a=$100\ \Omega$] ++ (0, 1.96)
            (amp.-) -- ++ (-2.75, 0)
            (amp.+) ++ (-2.75, 0) to[short, *-o] ++ (0, -0.5) node[below]{$In_+$}
            (amp.-) ++ (-2.75, 0) to[short, *-o] ++ (0, 0.5) node[above]{$In_-$}
            % (amp.+) ++ (-2.75, 0) to[short, *-o] ++ (-0.5, 0) node[left]{$V_{CM}$}

            (amp) ++ (1, -0.4) to[short, -o] ++ (0, -1) node[below]{$V_{REF+N}$}
            (amp.out) to[short, -o] ++ (1, 0) node[above]{$V_{out}$}

            % (amp.+) ++ (-1, -2) node[below]{$V_{REF+N}$} to[R, o-*, l=$R_1$, a=$10\ M\Omega$] ++ (0, 2) 
        ;
    \end{tikzpicture}
\end{subfigure}
}
        \caption{Schemat ideowy wzmacniacza pomiarowego.}
        \label{sch:INA333}
    \end{figure}
    \begin{figure}[!ht]
        \centering
        \scalebox{1}{\begin{subfigure}{\textwidth}
    \hspace{0cm}
    \begin{tikzpicture}
        \draw
            (14, 0) node[op amp](amp1){}
            (amp1.up) to[short, -o] ++ (0, 0.5) node[above]{$3.3\ V$}
            (amp1.down) node[ground]{}
            % (amp1.-) -| ++ (-0.5, 2) to[R, l=$R_2$, a=$100\ k\Omega$] ++ (2.88, 0) -- ++ (0, -2.5)
            (amp1.out) -- ++ (0, 2.5) to[R, l=$R_4$, a=$100\ k\Omega$, *-*] ++ (-2.5, 0) |- (amp1.-)

            (amp1.out) -- ++ (0, 4.5) to[C, l=$C_3$, a=$1\ \mu F$] ++ (-2.5, 0) |- (amp1.-)

            (amp1.-) to[short, -*] ++ (-0.12, 0)
            (amp1.out) to[short, *-o] ++ (1, 0) node[above]{$V_{REF+N}$}
            (amp1.-) to[C, l=$C_1$, a=$220\ nF$] ++ (-2, 0) to[R, l=$R_3$, a=$10\ k\Omega$, -*] ++ (-2, 0) ++ (-1.19, 0) node[op amp](amp2){}
            (amp2.up) to[short, -o] ++ (0, 0.5) node[above]{$3.3\ V$}
            (amp2.down) node[ground]{}
            (amp1.+) |-  ++ (-2, -1) -| (amp2.+)
            (amp2.out) -- ++ (0, 2.5) to[R, l=$R_2$, a=$100\ k\Omega$] ++ (-2.2, 0) -| (amp2.-)
            (amp2.-) to[R, l=$R_1$, a=$10\ k\Omega$, *-*] ++ (-2, 0) coordinate(Vin)

            (0, -1) node[ground]{} to[R, l=$R_6$, a=$10\ k\Omega$] ++ (0, 2) coordinate(Vref) 
            to[R, l=$R_5$, a=$10\ k\Omega$, *-o] ++ (0, 2) node[above]{$3.3\ V$}
            (Vref) to[short, *-*] ++ (2.5, 0) ++ (0, -2) node[ground]{} to[C, l=$C_2$, a=$1\ \mu F$] ++ (0, 2)
            (Vref) ++ (2.5, 0) -- (Vin) |- (amp2.+) to[short, *-] ++ (-0.1, 0)
        ;
    \end{tikzpicture}
\end{subfigure}}
        \caption{Schemat ideowy źródła napięcia referencyjnego z układem generacji szumu.}
        \label{sch:Vref_gen}
    \end{figure}
    % \begin{figure}[!ht]
    %     \centering
    %     \scalebox{1}{\begin{subfigure}{\textwidth}
    \hspace{5.5cm}
    \begin{tikzpicture}
        \draw
            (0, 0) node[op amp](amp1){}
            (amp1.up) to[short, -o] ++ (0, 0.5) node[above]{$3.3\ V$}
            (amp1.down) node[ground]{}
            (amp1.out) |-  ++ (-2, 2) -| (amp1.-)
            (amp1.+) to[short, -o] ++ (-1, 0) node[above]{$V_{CM}$}
            (amp1.out) to[short, *-o] ++ (1, 0) node[above]{$V_{CMOUT}$}
        ;
    \end{tikzpicture}
\end{subfigure}}
    %     \caption{Schemat do pomiaru napięcia sumacyjnego.}
    %     \label{sch:VCM_measure}
    % \end{figure}
    % \clearpage
    \begin{figure}[!ht]
        \centering
        \scalebox{1}{\begin{subfigure}{\textwidth}
    \hspace{4.5cm}
    \begin{tikzpicture}
        \draw
            (0, 0) node[op amp](amp1){}
            (amp1.up) to[short, -o] ++ (0, 0.5) node[above]{$3.3\ V$}
            (amp1.down) node[ground]{}
            (amp1.out) |-  ++ (-2, 2) -| (amp1.-)
            (amp1.+) to[short, -o] ++ (-1, 0) node[above]{$V_{IN}$}
            % (amp1.out) to[short, *-o] ++ (1, 0) node[above]{$V_{CMOUT}$}
            (amp1.out) to[R, l=$R_1$, l=$50\ \Omega$, *-] ++ (2, 0) coordinate(Vout) to[short, *-o] ++ (1, 0) node[above]{$V_{OUT}$}
            (Vout) to[C, l=$C_1$, a=$10\ nF$] ++ (0, -2) node[ground]{}
        ;
    \end{tikzpicture}
\end{subfigure}}
        \caption{Schemat pasywnego filtru RC o częstotliwości granicznej $f_g \approx 318\ kHz$.}
        \label{sch:LPF}
    \end{figure}
    % \begin{figure}[!ht]
    %     \centering
    %     \scalebox{1}{\begin{subfigure}{\textwidth}
    \hspace{3.5cm}
    \begin{tikzpicture}
        \draw
            (0, 0) node[ground]{} to[zDo, -*, l=$LM336$] ++ (0, 2) coordinate(out)
            to[R, -o, l=$R_1$, a=$820$] ++ (0, 2) node[above]{$3.3\ V$}
            (out) to[C, -o, l=$C_1$, a=$100\ nF$] ++ (3, 0) node[above]{$V_{out}$}
            (out) ++ (3.3, 0) node[circ, scale = 0.6]{}
            (out) ++ (3.5, 0) node[circ, scale = 0.6]{}
            (out) ++ (3.7, 0) node[circ, scale = 0.6]{}
            (out) ++ (5, 0) node[plain mono amp]{$?$}
        ;
    \end{tikzpicture}
\end{subfigure}}
    %     \caption{Schemat układu generacji szumu białego.}
    %     \label{sch:Noise_gen}
    % \end{figure}
    % \begin{figure}[!ht]
    %     \centering
    %     \scalebox{1}{\begin{subfigure}{\textwidth}
    \hspace{3cm}
    \begin{tikzpicture}
        \draw
            (0, 0) node[op amp, scale = 2](amp){}
            (amp.up) to[short, -o] ++ (0, 0.5) node[above]{$3.3\ V$}
            (amp.down) node[ground]{}
            
            (amp.+) to[R, l=$R_{4}$, a=$1.2k$] ++ (-2, 0) to[short, -o] ++ (0, 0.5) node[above]{$3.3\ V$} 
            (amp.+) ++ (0, -2.5) node[ground]{} to[R, -*, l=$R_{5}$, a=$1k$] ++ (0, 2.5)
            
            (amp.-) to[R, -o, l=$R_{3}$] ++ (-5.2, 0) node[above]{$V_{noise}$}

            (amp.out) -- ++ (0, 3) to[R, l=$R_{2}$, a=$10k$] ++ (-5, 0) coordinate(test)
            to[R, l=$R_{1}$, a=$10k$] ++ (-5, 0) to[short, -o] ++ (0, 0) node[above]{$V_{in}$}
            (test) to[short, -*] ++ (0, -2)
            (amp.out) to[short, -o] ++ (1, 0) node[above]{$V_{out}$}
        ;
    \end{tikzpicture}
\end{subfigure}}
    %     \caption{Układ sumacyjny dla dodania sygnału pomiarowego i szumu.}
    %     \label{sch:sumator}
    % \end{figure}
    % \begin{figure}[!ht]
    %     \centering
    %     \scalebox{1}{\begin{subfigure}{\textwidth}
    \hspace{4cm}
    \begin{tikzpicture}
    \draw
        (0, 0) node[draw, rectangle, minimum width = 3cm, minimum height = 3cm, label = {above:LT1117-3.3}](U1){}
        (U1.west) node[right=1mm] {IN}
        (U1.east) node[left=1mm] {OUT}
        (U1.south) node[above=1mm] {GND}
        (U1.south) node[ground]{}

        (U1.west) -- ++(-2,0) coordinate(IN)
        to[C, l=$C_{1}$, a=$10\ \mu$F] (IN |- U1.south) node[ground] {}
        (IN) to[short, -o] ++ (-1, 0) node[above]{$5V$}

        (U1.east) -- ++(2,0) coordinate(OUT)
        to[C, l=$C_{2}$, a=$100\ \mu$F] (OUT |- U1.south) node[ground] {}
        (OUT) to[short, -o] ++ (1, 0) node[above]{$3.3V$}

        ;
    \end{tikzpicture}
\end{subfigure}}
    %     \caption{Układ stabilizacji napięcia na 3.3V. }
    %     \label{sch:psu}
    % \end{figure}

    \begin{table}[!ht]
        \centering
        \begin{tabular}{|c|c|c|}\hline
            parametr & wartość & komentarz \\\hline
            czułość & $10\ \frac{mV}{\mu A}$ & czułość układu bez ditheringu \\\hline
            zakres pomiarowy & $\pm 100\ \mu A$ & - \\\hline
            zakres napięcia wspólnego & $0.7 \div 2.8\ V$ & - \\\hline
            częstotliwość graniczna & $30\ Hz$ & ograniczenie od INA333 \\\hline
            pasmo szumu białego & $100\ Hz \div 300\ kHz$ & \\\hline
            napięcie szumu & $\approx 1.5\ mV$ & w paśmie $100\ Hz \div 300\ kHz$ \\\hline
            % widmowa gęstość szumu & $\approx 1.65\ \frac{\mu V}{\sqrt{Hz}}$ & - \\\hline
            częstotliwość próbkowania & $f_S = 300\ kHz$ & z dokumentacji STM32F103C8T6 \\\hline
            częstotliwość nadpróbkowania & $f_{OVS} = 18.75\ kHz$ & nadpróbkowanie 16 razy \\\hline
            dodatkowe bity & $+2$ bity & na podstawie AN5537 \\\hline
            teoretyczna wartość mierzalnego & \multirow{2}{*}{$I_{meas} \approx 25\ nA$} & \multirow{2}{*}{jeśli nie tracimy na ENOB} \\ 
            prądu przy ditheringu  & & \\\hline
        \end{tabular}
        \caption{Parametry zaprojektowanego układu do pomiaru prądu.}
        \label{tab:params}
    \end{table}